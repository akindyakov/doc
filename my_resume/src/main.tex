\documentclass{article}

\usepackage [utf8] {inputenc}

\usepackage[unicode]{hyperref}

% включаем переносы для русского и английского языка
\usepackage[english,russian]{babel}

% Начинать первый параграф раздела следует с красной строки
\usepackage{indentfirst}

% Выбор внутренней TEX−кодировки
\usepackage [T2A]{fontenc}

\usepackage{cmap}

\usepackage{multirow}

\usepackage{geometry} % Меняем поля страницы
\geometry{left=2cm}% левое поле
\geometry{right=1.5cm}% правое поле
\geometry{top=2cm}% верхнее поле
\geometry{bottom=2cm}% нижнее поле


\begin{document}
{\Large\textbf{Александр Киндяков}}
% Линейка горизонтальная
\hrule \smallskip
% Линейка вертикальная нулевой толщины и нужной выстоты
\rule{0pt}{5mm}
{\normalsize
\itshape
%выравнивание по правому краю
\begin{flushright}
Госпитальный пер. 4/6 \\
г.Москва, Россия \\
тел: +7-926-453-25-30 \\
email: akindyakov@gmail.com \\
github: https://github.com/AKindyakov \\
\end{flushright}
}

{\large Цель } \rule{0pt}{1cm} \\

{\large Опыт работы } \rule{0pt}{1cm} \\
\hrule \smallskip
    \begin{tabular}{p{25mm}|p{110mm}}
    2011 - до настоящего момента
        & ``Лаборатория трехмерного зрения'' \\
        & Разработал систему автоматического управления 
            3х степенного манипулятора на основе микроконтролеров семейства STM8. \\
        & Занимался работой над системой навигации робота,
            написал алгоритм стыковки с докстанцией. \\
        & Поддержка ПО на платформу x86 для управления, 
            отладки и контроля за микроконтролерными системами на C++, Lua 5.1.  \\
        & Сделал несложную систему технического зрения (open CV) 
            для распознавания интерьеров по видеопотоку.  \\
    \end{tabular}

{\large Образование} \rule{0pt}{1cm} \\
\hrule \smallskip
    \begin{tabular}{p{25mm}|p{110mm}}
    2008 - до настоящего момента 
        & ``МГТУ им Н.Э. Баумана'' \\ 
        & Факультет ``Специальное машиностроение'' \\
        & Кафедра ``Специальная робототехника и мехатроника'' \\
        & URL: http://bstu.ru \\
        & Получаемая степень: Магистр \\
        & Год окончания 2014
    \end{tabular}

{\large Иностранные языки} \rule{0pt}{1cm} \\
\hrule \smallskip
Английский - Pre-Intermediate \\

%{\large Семейное положение} \rule{0pt}{1cm} &
%холост \\

{\large{Навыки}} \rule{0pt}{1cm} \\
\hrule \smallskip
    \begin{tabular}{ll}
    \textbullet \rule{2mm}{0pt} Языки программирования:  & C, C++, C/C++ Stl, Lua, Scilab, CMake \\
    \textbullet \rule{2mm}{0pt} Системы контроля версий: & Git, Subversion \\
    \textbullet \rule{2mm}{0pt} Операционные системы:    & GNU/Linux, Windows \\
    \end{tabular} \\

{\large{Общие сведения}} \rule{0pt}{1cm} \\
\hrule \smallskip
\begin{itemize}
    \item Занимался разработкой систем навигации роботов, систем стабилизации и
    распознования входного видеопотока. Некоторое время работал над искуственным
    интеллектом роботехнического комплекса. 
    \item Так же разрабатывал встроенные системы
    автоматического управления для манипуляторов. Занимался разработкой отладочного
    и настроечного кросплатформенного ПО для манипуляторов на C++, Boost и Lua 5.1. 
    \item Имею в качетсве хобби небольшой опыт разработки игр, используя C++, STL,
    Polycode Framework и cmake в качестве системы сборки.
    Знаком и имею небольшой опыт использования Python, использовал для своих целей
    библиотеку matplotlib. Для исследований в университете использовал Scilab.
    \item Во время работы в компании «Лаборатория трехмерного зрения» пользовался системой
    контроля версий subversion. Для личных проектов использую Git.
\end{itemize}


{\large{Интересы}} \rule{0pt}{1cm} \\
\hrule \smallskip
Алгоритмы и методики в области технического зрения, экпериметировал с OpenCV.
В последнее время интересуюсь реализацией классических алгоритмов,
разбираю книгу P. Седжвика, ``Алгоритмы C++''
Разработка игр \\


\end{document}

