\documentclass{article}

\usepackage [utf8] {inputenc}

\usepackage[unicode]{hyperref}

% включаем переносы для русского и английского языка
\usepackage[english,russian]{babel}

% Начинать первый параграф раздела следует с красной строки
\usepackage{indentfirst}

% Выбор внутренней TEX−кодировки
\usepackage [T2A]{fontenc}

\usepackage{cmap}

\usepackage{multirow}

\usepackage{geometry} % Меняем поля страницы
\geometry{left=2cm}% левое поле
\geometry{right=1.5cm}% правое поле
\geometry{top=2cm}% верхнее поле
\geometry{bottom=2cm}% нижнее поле


\begin{document}

{\Large\textbf{Александр Киндяков}}
% Линейка горизонтальная
\hrule \smallskip
% Линейка вертикальная нулевой толщины и нужной выстоты
\rule{0pt}{5mm}
{\itshape
    %выравнивание по правому краю
    \begin{flushright}
    Госпитальный пер. 4/6 \\
    г.Москва, Россия \\
    тел: +7-926-453-25-30 \\
    email: akindyakov@gmail.com \\
    github: https://github.com/AKindyakov \\
    \end{flushright}
}

{\large Цель } \rule{0pt}{1cm}

{\large Опыт работы } \rule{0pt}{1cm}
\hrule \smallskip
    \begin{tabular}{p{25mm}|p{110mm}}
    2011 - до настоящего момента
        & ``Лаборатория трехмерного зрения'' \\
        & Junior C / C++ Developer \\
        &   \begin{itemize}
            \item Разработал систему автоматического управления 
                3х степенного манипулятора на основе микроконтролеров семейства STM8.
            \item ПО для комплексного тестирования, настройки и диагностики
                неисправностей для узлов и систем манипуляторра робота.
            \item Занимался работой над системой навигации робота,
                написал алгоритм автоматической стыковки робота с докстанцией.
            \item Поддержка ПО на платформу x86 для управления, 
                отладки и контроля за микроконтролерными системами на C++, Lua 5.1.
            \item Сделал несложную систему технического зрения (open CV) 
                для распознавания интерьеров по видеопотоку.
            \item Учавствовал разработке систем стабилизации и
                распознования входного видеопотока.
            \item Работал над искуственным интеллектом роботехнического комплекса.
                (проект был заморожен) 
            \end{itemize}
        \\
    \end{tabular}

{\large Образование} \rule{0pt}{1cm}

\hrule \smallskip
    \begin{tabular}{p{25mm}|p{110mm}}
    2008 - до настоящего момента 
        & ``МГТУ им Н.Э. Баумана'' \\ 
        & Факультет ``Специальное машиностроение'' \\
        & Кафедра ``Специальная робототехника и мехатроника'' \\
        & URL: http://bstu.ru \\
        & Получаемая степень: Магистр \\
        & Год окончания 2014
    \end{tabular}

{\large Иностранные языки} \rule{0pt}{1cm}
\hrule \smallskip
Английский - Pre-Intermediate

{\large{Навыки}} \rule{0pt}{1cm}

\hrule \smallskip
\begin{itemize}
    \item Языки программирования:   C, C++, C/C++ STL, Lua, Scilab, CMake
    \item Системы контроля версий:  Git, Subversion
    \item Операционные системы:     GNU/Linux, Windows
\end{itemize}

{\large{Общие сведения}} \rule{0pt}{1cm}
\hrule \smallskip
\begin{itemize}
    \item Семейное положение: холост
    \item В качетсве хобби разрабатываем с другом небольшую игру, в которой
        используем C++, STL, Polycode Framework и CMake в качестве системы сборки.
    \item Знаком и имею небольшой опыт использования Python, использовал для своих целей
        библиотеку matplotlib. Для исследований в университете использовал Scilab.
    \item Во время работы в компании ``Лаборатория трехмерного зрения'' пользовался системой
        контроля версий subversion. Для личных проектов использую Git.
\end{itemize}

\rule{0pt}{1cm}
{\large{Интересы}}
\hrule \smallskip
\begin{itemize}
    \item Алгоритмы и методики в области технического зрения, экпериметировал с OpenCV.
    \item В последнее время интересуюсь реализацией классических алгоритмов,
        разбираю книгу P. Седжвика, ``Алгоритмы C++''
    \item Разработка игр 
    \item Литература, 
\end{itemize}

\end{document}

