\documentclass{article}

\usepackage [utf8] {inputenc}

\usepackage[unicode]{hyperref}

% включаем переносы для русского и английского языка
\usepackage[english,russian]{babel}

% Начинать первый параграф раздела следует с красной строки
\usepackage{indentfirst}

% Выбор внутренней TEX−кодировки
\usepackage [T2A]{fontenc}

\usepackage{cmap}

\usepackage{multirow}

\usepackage{geometry} % Меняем поля страницы
\geometry{left=2cm}% левое поле
\geometry{right=1.5cm}% правое поле
\geometry{top=2cm}% верхнее поле
\geometry{bottom=2cm}% нижнее поле


\begin{document}

{\Large\textbf{Александр Киндяков}}\\
{\large\textbf{C/C++ разработчик}}
% Линейка горизонтальная

% Линейка вертикальная нулевой толщины и нужной выстоты
\rule{0pt}{5mm}
{\itshape
    %выравнивание по правому краю
    \begin{flushright}
    Госпитальный пер. 4/6 \\
    г.Москва, Россия \\
    +7-926-453-25-30 \\
    \textbf{EMail:} akindyakov@gmail.com \\
    \textbf{GitHub:} https://github.com/AKindyakov \\
    \textbf{LinkedIn:} http://ru.linkedin.com/in/alexanderkindyakov/ \\
    \end{flushright}
}

\rule{0pt}{1cm}
{\large{Общие сведения}}
\hrule \smallskip
\begin{itemize}
    \item Опыт разработки приложений на С++ по работе с графикой, оработке
    изображений и видеопотока, работе с периферийным оборудованием. 
    Работал с библиотеками Boost, OpenCV. 
    \item Некоторое время работал над искуственным интеллектом роботехнического комплекса.
    \item Так же разрабатывал встроенные системы автоматического управления для
    манипуляторов. 
    \item Занимался разработкой отладочного и настроечного
    кросплатформенного ПО (Ubuntu, Windows) для манипуляторов на C++, Boost и Lua 5.1. 
    \item Занимался разработкой систем навигации для мобильных платформ.
    \item Опыт участия в OpenSource проекте SciLab.
    \item Имею в качетсве хобби небольшой опыт разработки игр, используя C++, STL,
    Polycode Framework, в качестве систем сборки использовал cmake.
\end{itemize}

\rule{0pt}{1cm}
{\large Опыт работы }
\hrule \smallskip
    \begin{tabular}{p{25mm}|p{110mm}}
    2011 - до настоящего момента
        & ``Лаборатория трехмерного зрения'' \\
        & Junior C / C++ Developer \\
        &   \begin{itemize}
            \item Разработал систему автоматического управления 
                3х степенного манипулятора на основе микроконтролеров семейства STM8.
            \item ПО для комплексного тестирования, настройки и диагностики
                неисправностей для узлов и систем манипуляторра робота.
            \item Занимался работой над системой навигации робота,
                написал алгоритм автоматической стыковки робота с докстанцией.
            \item Поддержка ПО на платформу x86 для управления, 
                отладки и контроля за микроконтролерными системами на C++, Lua 5.1.
            \item Сделал несложную систему технического зрения (open CV) 
                для распознавания интерьеров по видеопотоку.
            \item Учавствовал разработке систем стабилизации и
                распознования входного видеопотока.
            \item Работал над искуственным интеллектом роботехнического комплекса.
                (проект был заморожен) 
            \end{itemize}
        \\
    \end{tabular}

\rule{0pt}{10mm}
{\large Образование}
\hrule \smallskip
    \begin{tabular}{p{25mm}|p{110mm}}
    2008 - до настоящего момента 
        & ``МГТУ им Н.Э. Баумана'' \\ 
        & Факультет ``Специальное машиностроение'' \\
        & Кафедра ``Специальная робототехника и мехатроника'' \\
        & URL: http://bstu.ru \\
        & Получаемая степень: Магистр \\
        & Год окончания 2014
    \end{tabular}

\rule{0pt}{1cm}
{\large Иностранные языки}
\hrule \smallskip
\begin{itemize}
    \item Русский    - родной язык
    \item Английский - Pre-Intermediate
\end{itemize}

\rule{0pt}{1cm}
{\large{Навыки}}
\hrule \smallskip
\begin{itemize}
    \item Языки программирования:   C, C++, Lua(5.1, 5.2), CMake, Shell. Начал знакомство с Python
    \item Библиотеки:               C/C++ STL, OpenCV, Boost
    \item Прикладные программы:     CMake, make, Vim, doxygen, SciLab, Matlab
    \item Системы контроля версий:  Git, Subversion
    \item Операционные системы:     GNU/Linux, Windows
\end{itemize}

\rule{0pt}{1cm}
{\large{Научные и фундаментальные знания}}
\hrule \smallskip
\begin{itemize}
    \item Системы автоматического управления
    \item Алгоритмы и структуры данных, линейная алгебра, дискретная математика
    \item Теоретическая механика
\end{itemize}

\rule{0pt}{1cm}
{\large{Интересы}}
\hrule \smallskip
\begin{itemize}
    \item Алгоритмы и методики в области технического зрения, экпериметировал с OpenCV.
    \item Классические алгоритмы и ихреализация, разбираю книгу P. Седжвика, ``Алгоритмы C++''
    \item Литература: читаю книги, как о разработке и проектировании програмных
продуктов так и художественные. 

http://www.goodreads.com/user/show/24404721-alexander-kindyakov    
    \item Разработка игр: в компании с моим другом разрабатываем с другом небольшую игру, в которой
используем C++, STL, Polycode Framework и CMake в качестве системы сборки.
 
https://github.com/TheGreenBox/helium 
    \item Спорт: бальные танцы и плавание
\end{itemize}

\end{document}

